%%%%%%%%%%%%%%%%%%%%%%% file template.tex %%%%%%%%%%%%%%%%%%%%%%%%%
%
% This is a general template file for the LaTeX package SVJour3
% for Springer journals.          Springer Heidelberg 2010/09/16
%
% Copy it to a new file with a new name and use it as the basis
% for your article. Delete % signs as needed.
%
% This template includes a few options for different layouts and
% content for various journals. Please consult a previous issue of
% your journal as needed.
%
%%%%%%%%%%%%%%%%%%%%%%%%%%%%%%%%%%%%%%%%%%%%%%%%%%%%%%%%%%%%%%%%%%%
\RequirePackage{fix-cm}
\documentclass[smallextended, 12pt]{article}  
\addtolength{\hoffset}{-2cm}
\addtolength{\textwidth}{3.5cm}
\addtolength{\voffset}{-2cm}
\addtolength{\textheight}{5cm}    
\usepackage{url}
% please place your own definitions here and don't use \def but
% \newcommand{}{}
%
% Insert the name of "your journal" with
% \journalname{myjournal}
%
\begin{document}
	\title{Alzheimer}
	\author{\large Lopez Flores Ana Laura}
	\maketitle
	
	\begin{abstract}
	El 21 de septiembre de cada a\~no se celebra el D\'ia Mundial de la enfermedad de Alzheimer, la causa m\'as frecuente de demencia neurodegenerativa en las personas mayores de 65 a\~nos. El Alzheimer produce un deterioro progresivo y total de las funciones cognitivas (p\'erdida de memoria, alteraci\'on del lenguaje, p\'erdida del sentido de la orientaci\'on y de las funciones ejecutivas), frecuentemente acompa\~nada de cambios en la personalidad y en el comportamiento, y que conlleva una dram\'atica reducci\'on de la capacidad del individuo para llevar a cabo las actividades de su vida diaria. Actualmente no se conocen las causas que originan esta devastadora enfermedad, motivo principal por el que a\'un no existe prevenci\'on ni cura.
	\\
	\\
	En un sentido estricto, el Alzheimer s\'olo puede ser confirmado mediante el an\'alisis post-mortem del tejido cerebral, por lo que el desarrollo y validaci\'on de t\'ecnicas de diagn\'ostico precoz es otro de los grandes retos de la investigaci\'on actual, ya que hoy en d\'ia el diagn\'ostico del Alzheimer se realiza cuando la enfermedad ha causado un extenso e irreparable da\~no cerebral siendo as\'i imposible el tratamiento eficaz de esta.
	\end{abstract}
	
	\section{Introducci\'on}
	La \textbf{demencia} es uno de los trastornos cerebrales org\'anicos m\'as importantes. Se manifiesta de forma cr\'onica y progresiva, con presencia de alteraciones en funciones cognitivas como la memoria, el pensamiento, la orientaci\'on, el c\'alculo, el lenguaje y la capacidad de aprendizaje sin tener que producir, en principio, un trastorno en la conciencia. Sin embargo, puede ocasionar un deterioro en el control emocional, en el comportamiento social o en la motivaci\'on. El deterioro producido por la demencia se refleja en la actividad diaria del enfermo, por ejemplo, en el aseo personal, en el vestirse, en el comer o en las funciones excretoras. \cite{ref3}\\
	\\
	Las demencias pueden clasificarse atendiendo a m\'ultiples criterios, uno de los m\'as utilizados es en funci\'on del grado de reversibilidad:
	\begin{itemize}
		\item \textit{Reversibles}, que pueden mejorar a trav\'es de un tratamiento o de una operaci\'on quir\'urgica.
		\item \textit{Irreversibles}, en las que no hay ning\'un tipo de tratamiento, ya que son causadas por una lesi\'on cerebral. Sin embargo, si pueden haber medicamentos que ayuden a que la enfermedad avance m\'as lentamente. \cite{ref3}
	\end{itemize} 	
	Ahora bien, la enfermedad de Alzheimer (EA) es el tipo de demencia irreversible m\'as frecuente en la vejez. Es una enfermedad neurodegenerativa que se manifiesta como deterioro cognitivo y trastornos conductuales. Se caracteriza, en su forma t\'ipica, por una p\'erdida de la memoria inmediata y de otras capacidades mentales a medida que mueren las c\'elulas nerviosas (neuronas) y se atrofian diferentes zonas del cerebro. La enfermedad suele tener una duraci\'on media aproximada despu\'es del diagn\'ostico de 10 a\~nos, aunque esto puede variar en proporci\'on directa con la severidad de la enfermedad al momento del diagn\'ostico. Se le considera una enfermedad social, pues sus repercusiones no son solamente en los pacientes sino en el entorno familiar. \cite{ref3} \cite{ref4}
		
	
	\section{Justificaci\'on}
	La presente investigaci\'on se enfocara en la b\'usqueda de informaci\'on relacionada a los principales s\'intomas de la enfermedad de Alzheimer as\'i como de los tratamientos existentes de este terrible mal; debido al alto porcentaje de personas que la padecen, las graves consecuencias se sabe puede acarrear el padecer Alzheimer y por sobre todo lo tard\'io que llega a ser su diagn\'ostico, me parece de relevancia conocer lo ya dicho para que pueda ser tomado en cuenta si se llegase a presentar la ocasi\'on. \\
	\\
	Esta investigaci\'on se realizara principalmente para un bien informativo en el que se busca que el interesado en leerla obtenga datos relevantes y verdaderos que se pretende le sirvan de utilidad para aclarar a grandes rasgos las dudas que se tengan respecto al Alzheimer y principalmente sobre su diagn\'ostico y tratamiento efectivo. \\
	\\
	El uso que se le d\'e a la informaci\'on expuesta en la presente depender\'a \'unica y exclusivamente del lector, aunque se espera que entienda los aspectos generales acerca de esta enfermedad as\'i como de los temas antes mencionados. \cite{ref1} \cite{ref2}
	
	\section{Objetivos generales}
	Se busca investigar cuales son los s\'intomas m\'as representativos de la enfermedad de Alzheimer as\'i como los principales tratamientos existentes para el control de Alzheimer y los avances relacionados al tema que han acontecido.
	\subsection{Objetivos espec\'ificos}
	\begin{itemize}
		\item Causas de la enfermedad.
		\item Principales s\'intomas de la enfermedad.
		\item Formas de detecci\'on/diagn\'ostico de Alzheimer.
		\item Tratamientos com\'unmente utilizados para tratarla.\\
	\end{itemize} 
	
	\section{Desarrollo}
	\subsection{Causas de Alzheimer}
	A pesar de los innumerables estudios que se han hecho sobre la enfermedad, no se han llegado a encontrar causas o razones espec\'ificas por las que el Alzheimer aparece; se desconoce cu\'al es el motivo de su aparici\'on, porqu\'e unas personas la padecen y otras no, m\'as all\'a del hecho de ser por la degeneraci\'on de las c\'elulas del cerebro. Cient\'ificos y doctores especulan que no se llegar\'a a encontrar una raz\'on o factor inherente en todos los casos de Alzheimer y que esta enfermedad es consecuencia de varios factores interrelacionados en la gen\'etica y estilo de vida de una persona. Algunos de estos factores son: \cite{ref4}
	\begin{itemize}
		\item Tener un pariente consangu\'ineo cercano, como hermanos o padres, que tengan la enfermedad. Hay que a\~nadir que no necesariamente es hereditaria, es decir, la herencia  es un factor que puede influir en que aparezca la enfermedad, pero que no necesariamente va a determinar su aparici\'on. \cite{ref4}\cite{ref3}
		\item Tener ciertos genes ligados al mal de Alzheimer como el alelo APOE epsilon4. 
		\item Tener presi\'on arterial alta.
		\item Niveles altos de colesterol.
		\item Antecedentes de traumatismo craneal.
		\item Niveles de educaci\'on.
		\item Traumas psicol\'ogicos en la infancia. \cite{ref4}
	\end{itemize} 
	Por otro lado, se sabe con certeza que el hecho de ser mayor no quiere decir que vaya a desarrollar la enfermedad, es decir, no forma parte de un envejecimiento normal aunque sea m\'as frecuente en personas de edad avanzada (la mayor\'ia de las personas que la padecen tienen m\'as de 65 a\~nos). Las personas pueden desarrollar los s\'intomas de esta enfermedad en cualquier momento desde los 40 a\~nos, aunque lo m\'as com\'un es que empiece a mediados de los 60. \cite{ref3} \\
	\\
	Como ya se ha mencionado, aunque no existen causas espec\'ificas para esta enfermedad los avances tecnol\'ogicos de hoy en d\'ia han permitido que se pueda entender de mejor manera qu\'e es lo que pasa dentro del cerebro con el mal de Alzheimer. \cite{ref4}\\
	\\
	Se est\'a al tanto de que el desarrollo de placas y ovillos en la estructura del cerebro lleva a la muerte de las neuronas. Los pacientes con EA tambi\'en tienen deficiencia de algunos neurotransmisores en el cerebro. Es una enfermedad progresiva, por lo que cuantas m\'as partes del cerebro se da\~nan, los s\'intomas se vuelven m\'as graves. \cite{ref5}\\
	\\
	Gracias a las nuevas tecnolog\'ias existentes se han identificado cuatro genes que influencian el desarrollo de la enfermedad: el gen de la prote\'ina precursora amiloide (APP, en el cromosoma 21) y dos genes de presenilina (PS) (PS1 y PS2, en los cromosomas 14 y 1, respectivamente) son causantes de la forma familiar. Las personas con cualquiera de estos genes tienden a desarrollar la enfermedad entre los 30 y 40 a\~nos y vienen de familias en las que varios miembros tambi\'en tienen EA de aparici\'on temprana. Suman menos de 1 en cada 1.000 casos. Sin embargo, los portadores de la variante gen\'etica ApoE4 (ubicado en el cromosoma 19), que transporta el colesterol por la corriente sangu\'inea de forma incorrecta causando problemas cardiovasculares, tienen mayores posibilidades de desarrollar la enfermedad. La EA de aparici\'on tard\'ia o espor\'adica ocurre alrededor de los 65 a\~nos y es la forma m\'as com\'un de la enfermedad. El gen de la apolipoprote\'ina E (ApoE) tiene efectos que aparecen m\'as sutilmente que los de los otros genes de aparici\'on temprana, e incluso los individuos con dos copias de la forma de riesgo del gen no necesariamente desarrollar\'an la enfermedad. \cite{ref6} \cite{ref5}
	
	\subsection{S\'intomas}
	Uno de los mayores problemas que existen con Alzheimer es que los s\'intomas tempranos de la enfermad pueden confundirse con procesos normales de envejecimiento, generalmente los familiares indican que los estos son despistes u olvidos a los que no dan importancia porque se cree parte de la senilidad. Es por esta raz\'on que es necesario crear conciencia sobre los primeros indicios de Alzheimer para poder tratar y aconsejar a una persona y sus familiares sobre la enfermedad*. En cada persona la enfermedad aparece y evoluciona de una manera distinta, pero generalmente comienza con dificultad para: \cite{ref4} \cite{ref3}
	\begin{itemize}
		\item Recordar hechos recientes, por lo que suelen preguntar varias veces lo mismo.
		\item Aprender cosas nuevas. 
		\item Adaptarse a nuevas situaciones.
		\item Expresar sus emociones o sentimientos.
		\item Mantener una conversaci\'on, para utilizar las palabras correctamente o dificultad para encontrarlas por lo que utilizan con frecuencia los determinantes ``esto'', ``eso'', ``aquello'' como sustitutivo.
		\item Manejar dinero; \'este va perdiendo su ``valor'' 
		\item Tomar de decisiones. \cite{ref3}
	\end{itemize} 
	Cabe mencionar que los s\'intomas iniciales no van apareciendo todos a la vez sino poco a poco y progresivamente aumentan en frecuencia e intensidad, a\~nadi\'endosele nuevas alteraciones o dificultades en otras \'areas y capacidades como lo pueden ser: \cite{ref3}
	\begin{itemize}
		\item  Dificultad al realizar tareas que resultaban familiares, tales como cocinar, lavar platos, abrir puertas, manejar electrodom\'esticos, etc.
		\item \textit{Afasia:} La p\'erdida de poder producir o comprender el lenguaje. Utilizar palabras sin sentido en ciertas frases, falta de fluidez al hablar y p\'erdida en la capacidad de leer y escribir.
		\item P\'erdida de criterio que hace que el paciente tenga comportamientos incoherentes, como tener ciertas actitudes en p\'ublico que normalmente no tendr\'ia. 
		\item  Perder noci\'on de tiempo y espacio, no saber qu\'e a\~no o fecha es, perderse en lugares que antes resultaban familiares, confundir lugares con otros que exist\'ian en su pasado. 
		\item 	Cambios en la personalidad, que suelen ser violentos y dram\'aticos mostrando facetas de una persona que son completamente ajenas a esta. \cite{ref4}
	\end{itemize} 
	*(En el \textbf{Anexo 1} se podr\'a encontrar una tabla comparativa sobre los principales s\'intomas de Alzheimer y cambios t\'ipicos relacionados con la edad) \cite{ref7}
	
	\subsection{Etapas de la enfermedad}
	La rapidez de la progresi\'on de la enfermedad var\'ia considerablemente. Las personas que padecen de Alzheimer viven un promedio de ocho a\~nos, pero algunas personas pueden vivir hasta 20 a\~nos. El ciclo de la enfermedad depende en parte de la edad de la persona al momento del diagn\'ostico y de la existencia de otras condiciones m\'edicas. Se han identificado tres etapas de acuerdo con la evoluci\'on de la enfermedad: temprana, intermedia y final. Sin embargo, se debe tener en cuenta que en cada persona los signos o s\'intomas se pueden presentar en forma diferente. \cite{ref3} \cite{ref8}
	
	\subsubsection{\textit{Etapa temprana (Fase I)}}
	La etapa temprana se manifiesta de forma insidiosa, lenta y progresiva. Se caracteriza por presentar cortos per\'iodos de p\'erdida de la memoria, torn\'andose olvidadiza, con dificultades para concentrarse y para recordar las palabras correctas. Le cuesta trabajo adaptarse a cambios en sus viejas rutinas y poco a poco la mirada del enfermo se hace inexpresiva, ``perdida'‘. \cite{ref3} \cite{ref8} \\
	\\
	En su comportamiento se aprecian dificultades para realizar tareas complejas. Se les dificulta manejar sus finanzas, o planificar tareas de hogar complejas. En el trabajo la capacidad de planificar acciones dirigidas hacia un objetivo se ven limitadas, as\'i como su capacidad para brindar informes verbales o escritos. El aspecto personal empieza a fallar y requiere de ayuda para seleccionar apropiadamente las prendas de vestir. Durante esta etapa las personas reconocen que algo anda mal, se dan cuenta de su p\'erdida de memoria, y pueden sentirse asustadas o confundidas por los cambios, por lo que intentan esconder el problema o justificar los cambios de comportamiento. \cite{ref8}\\
	\\
	En esta etapa es dif\'icil hacer un diagn\'ostico preciso de la enfermedad, pues se presenta antes de que los s\'intomas puedan ser detectados por pruebas actuales. Sin embargo, las placas y las mara\~nas empiezan a formarse en las \'areas del cerebro que son para: aprender y recordar o pensar y planear. \cite{ref3} \cite{ref8}
	
	\subsubsection{\textit{Etapa intermedia (Fase II) }}
	En la etapa intermedia el comportamiento de la persona se deteriora notablemente y la p\'erdida de la memoria (reciente y de evocaci\'on.) es m\'as severa. Dura de 2 a 10 a\~nos. Muchas veces se le dificulta reconocer a familiares y amigos, y est\'a menos dispuesta para aprender a adaptarse a nuevas situaciones. La persona podr\'ia tener bruscos cambios de personalidad, torn\'andose muy activa, realizando movimientos continuos y repetitivos (mover los pies, chuparse los labios, etc), caminar sin parar e incluso llegar a deambular sin rumbo por las calles, siendo incapaces de regresar a casa. \cite{ref3} \cite{ref8}\\
	\\
	Se pierde tambi\'en el equilibrio y se modifican y retardan los movimientos. La creciente confusi\'on hace que cada vez resulte m\'as dif\'icil enfrentarse a la vida diaria, pierden la capacidad para recordar normas b\'asicas de seguridad, por lo tanto actividades que sab\'ian hacer como cocinar, conducir un auto, manejar herramientas, se tornan peligrosas. en pocas palabras se pierde autonom\'ia.	Durante esta etapa las personas pueden necesitar asistencia para desarrollar algunas actividades de la vida diaria como ba\~narse, vestirse, comer, etc. Pueden presentar un aumento notable en el apetito y la ingesti\'on de comida, desasosiego y trastornos del sue\~no. \cite{ref3} \cite{ref8}\\
	\\
	Las regiones del cerebro que son importantes para la memoria, hablar y comprender el habla y entender la posici\'on de su cuerpo en relaci\'on a los objetos a su alrededor desarrollan m\'as placas y mara\~nas de las que estuvieron presentes en las etapas tempranas. Muchas personas con la enfermedad de Alzheimer son diagnosticadas durante estas etapas. \cite{ref3}
	
	\subsubsection{\textit{Etapa final (Fase III)}}
	En la etapa final los pacientes se ven severamente desorientados, no reconocen a las personas ni los lugares, ni saben en qu\'e d\'ia est\'an. Hay un deterioro agudo de su capacidad motriz, que progresa de una incapacidad para caminar a una incapacidad para levantarse. Las personas pueden terminar postradas en una cama y llegar a perder el control de esf\'interes, siendo incapaces de controlar la emisi\'on de orina y de materia fecal. Puede durar de 1 a 5 a\~nos. \cite{ref3} \cite{ref8}\\
	\\
	En esta fase las personas presentan una p\'erdida en la capacidad de expresi\'on verbal y no verbal, no se sabe si est\'an tristes o felices. Se convierten en personas muy silenciosas y retra\'idas. A medida que progresa la enfermedad, la reacci\'on a los est\'imulos que las rodean contin\'ua descendiendo, hasta que dejan de reaccionar totalmente. \cite{ref8}\\
	\\
	La mayor\'ia de la corteza est\'a seriamente da\~nada, el cerebro se encoge dram\'aticamente debido a la muerte de un gran n\'umero de c\'elulas. En esta fase se desarrolla una alta propensi\'on a todo tipo de enfermedades e infecciones, ya que la persona ha ido perdiendo casi todas sus defensas. La persona con esta enfermedad puede durar varios a\~nos, y generalmente llega a la muerte por otras causas, como son las infecciones, accidentes, mal nutrici\'on, etc. \cite{ref3} \cite{ref8}
	
	\subsection{Detecci\'on/Diagn\'ostico}
	Al inicio de la enfermedad de Alzheimer, diversos estados patol\'ogicos, presentan s\'intomas similares a los que presenta la enfermedad de Alzheimer, por esto es necesario conocer las caracter\'isticas de cada uno y recurrir a diferentes ex\'amenes, es imprescindible tener en cuenta los criterios de los distintos diagn\'osticos diferenciales para no conducir a la realizaci\'on de un diagn\'ostico err\'oneo. Algunas de estas enfermedades son: \cite{ref6}
	\begin{itemize}
		\item Demencia vascular
		\item La demencia de la enfermedad de Parkinson
		\item Demencia con cuerpos de Lewy
		\item La enfermedad de Creutzfeldt-Jakob
		\item La hidrocefalia de presi\'on normal \cite{ref9} 
	\end{itemize} 
	(Para m\'as informaci\'on ver el \textbf{Anexo 2}: Tipos de Demencias) \cite{ref9}\\
	\\
	Tambi\'en puede confundirse con diferentes condiciones como lo son infecciones, deficiencia vitam\'inica, problemas tiroideos, tumores cerebrales, efectos secundarios de f\'armacos, diferentes formas de intoxicaci\'on y depresi\'on. Adem\'as de evaluar la memoria y habilidades cognitivas, puede realizarse una tomograf\'ia computarizada o una resonancia magn\'etica pues esto puede revelan anomal\'ias en la parte medial del l\'obulo temporal y en los casos avanzados es posible observar la corteza cerebral atr\'ofica adelgazada y ventr\'iculos laterales dilatados. A medida que la enfermedad progresa, su desarrollo caracter\'istico, permite establecer el diagnostico confiable con mayor facilidad. Actualmente la \'unica manera definitiva para diagnosticar la EA es investigar sobre la existencia de placas y ovillos en el tejido cerebral. Pero para observar el tejido cerebral se debe esperar hasta que se haga la autopsia. Por tanto, el diagn\'ostico de EA es cl\'inico, no existen test de laboratorio que confirmen la presencia de la enfermedad, excepto post-mortem. \cite{ref5} \cite{ref6} \cite{ref3}

	\subsection{Tratamiento}
	En la actualidad, la enfermedad de Alzheimer no tiene una cura conocida, no existen medicamentos o terapias espec\'ificas para detener o retardar su proceso. Sin embargo, un manejo cuidadoso puede ayudar a disminuir en gran medida el sufrimiento del paciente. Esta enfermedad requiere cuidado intensivo y profesional durante 8 a 10 a\~nos y se convierte en una gran carga para la familia, la cual juega un papel principal en el tratamiento y necesita gran apoyo. \cite{ref8} \cite{ref6}\\
	\\
	Aunque no existe cura para la enfermedad, se encuentran disponibles algunos tratamientos que pueden aminorar los s\'intomas. Com\'unmente se dividen los s\'intomas de la EA en cognitivos, de comportamiento y psiqui\'atricos. Los cognitivos afectan a la memoria, el lenguaje, el juicio, la habilidad para prestar atenci\'on y otros procesos del pensamiento, mientras que los s\'intomas de comportamiento y psiqui\'atricos afectan a la forma en que nos sentimos y actuamos. \cite{ref5}\\
	\\
	Hay dos tipos de medicamentos aprobados para el tratamiento de los s\'intomas cognitivos de la EA: los inhibidores de la colinesterasa y la memantina. Los inhibidores de la colinesterasa apoyan la comunicaci\'on entre c\'elulas nerviosas, manteniendo niveles altos de acetilcolina. La memantina regula la actividad del glutamato y previene la entrada excesiva de iones calcio en las c\'elulas del cerebro. \cite{ref5}\\
	\\
	En diferentes etapas de la EA, los pacientes pueden experimentar problemas f\'isicos o verbales, angustia, intranquilidad, alucinaciones y delirio. Existen dos formas de controlar los s\'intomas psiqui\'atricos: utilizando medicamentos para controlar espec\'ificamente los cambios en el comportamiento o estrategias sin f\'armacos; estas \'ultimas siempre deben intentarse primero e incluyen cambiar el ambiente de la persona para proporcionarle comodidad, seguridad y tranquilidad. El tratamiento no farmacol\'ogico desprende el empleo de una serie de t\'ecnicas que sirven para estimular cognitiva y afectivamente al enfermo entre las que se encuentran:
		\begin{itemize}
			\item Entrenamiento de memoria. Adivinanzas, refranes, ejercicios de completar frases, describir objetos, reconocer personas, asociar parejas, entre otros.
			\item T\'ecnicas de orientaci\'on a la realidad. Repetirle diariamente, pero sin agobiarle, todos los datos correctos sobre los lugares, la fecha, la familia, el lugar donde est\'an las cosas, etc.
			\item Reminiscencias. Resulta muy provechoso que narre historias, que cuente cosas del pasado, que hable de todo lo que recuerde. \cite{ref3}
		\end{itemize} 

	Como se puede ver ninguna droga contribuye a revertir la enfermedad, lo que en realidad se hace es tratar algunos de los problemas que se van presentando a lo largo de la enfermedad, por ello los medicamentos deben tener como objetivo s\'intomas espec\'ificos para que sus efectos puedan ser monitorizados. Por ejemplo, se utilizan medicamentos para aliviar la depresi\'on o la ansiedad, para disminuir el comportamiento agresivo, o para tratar las infecciones u otras alteraciones org\'anicas que afecten al paciente. Por esto es importante que la familia o la persona encargada del cuidado directo del paciente haga una observaci\'on cuidadosa del comportamiento, de los signos y s\'intomas que presente, e informe de cualquier cambio que observe relacionados con su salud f\'isica o mental. \cite{ref8}\\
	\\
	Adem\'as de lo anterior, recientemente se ha venido viendo con un poco m\'as de \'enfasis la prevenci\'on aut\'enticamente primaria, es decir, la prevenci\'on en personas que ni presentan s\'intomas, ni marcadores biol\'ogicos, ni se sabe si alguna vez desarrollaran o no la enfermedad.  Esto debido que, al no disponerse de un tratamiento realmente eficaz en este momento, se est\'an realizando varios estudios con nuevos medicamento que si bien pueden ayudar a retrasar su evoluci\'on lo m\'as probable es que solo sean efectivos en un porcentaje muy bajo de la poblaci\'on afectada. Por este motivo y por ahora, se cree que el futuro est\'a en la prevenci\'on primaria, en llevar una vida saludable para ayudar a disminuir la posibilidad de padecer de Alzheimer.\cite{ref10}
	
	\section{Conclusiones}
	El diagn\'ostico del Alzheimer se basa en la actualidad en una descripci\'on detallada del comportamiento del paciente, junto con la realizaci\'on de un examen del estado f\'isico y neurol\'ogico, as\'i como de distintas pruebas complementarias que permiten descartar la existencia de tumores cerebrales adem\'as de otras causas potencialmente tratables. \\
	\\
	Hoy se sabe que las lesiones cerebrales ocurren de 10 a 20 a\~nos antes de que se manifiesten los primeros s\'intomas cl\'inicos, por lo que es prioritario encontrar biomarcadores de la evoluci\'on de la enfermedad que permitan detectarla en fases tempranas en las que a\'un no se presentan sus s\'intomas, y antes de que se haya producido una importante p\'erdida de sinapsis y neuronas, cuando las estrategias terap\'euticas en desarrollo tendr\'ian mayores posibilidades de \'exito. \\
	\\
	Existe una falta de concientizaci\'on sobre el Alzheimer y muchas veces se confunde con procesos normales de envejecimiento, lo que lleva a falta de diagn\'ostico y tratamiento. Por ello, es fundamental para el paciente y/o los familiares y allegados a estos prestar gran atenci\'on a los se\~nales de padecimiento de Alzheimer, estar alerta a los posibles indicadores de esta ayuda en gran medida a su diagn\'ostico precoz as\'i como su tratamiento temprano. Si se llegase a presentar alguno de los sus s\'intomas es recomendable ir a realizarse los ex\'amenes correspondientes.\\
	\\
	La \'unica v\'ia posible para frenar el avance vertiginoso de esta enfermedad y reducir el n\'umero de afectados en un futuro pr\'oximo es la investigaci\'on.	
	
	\section{Trabajo futuro}
	Hoy d\'ia la cura para el Alzheimer no existe y es bien sabido sobre la existencia de diversas organizaciones que se dedican \'unica y exclusivamente a la investigaci\'on de esta y de medios para su detecci\'on en sus primeras etapas, donde su tratamiento aun es efectivo. Por lo que es indiscutible el trabajo futuro que el tema comprende.
	
	\section{Anexos}
	\subsection{Anexo 1: 10 se\~nales de advertencia de la enfermedad de Alzheimer}
	La Alzheimer'’s Association ha creado esta lista de se\~nales de advertencia de la enfermedad de Alzheimer y otros tipos de demencia. Cada individuo puede experimentar una o m\'as de estas se\~nales a grados diferentes. Si se es notada cualquiera de ellas, es recomendable consultar a un m\'edico. \cite{ref7}
	\begin{center}
		\begin{tabular}{|p{10cm}|p{7cm}|}
			\hline
			\textbf{S\'intomas de Alzheimer }&\textbf{ Cambios t\'ipicos relacionados con la edad} \\
			\hline
			
			Una de las se\~nales m\'as comunes es olvidar informaci\'on reci\'en aprendida. Se olvidan fechas o eventos importantes; se pide la misma informaci\'on repetidamente; se depende en sistemas de ayuda para la memoria o en familiares para hacer las cosas que antes uno se hac\'ia solo. & Olvidarse de vez en cuando de nombres o citas pero acord\'andose de ellos despu\'es. \\ \hline
			Algunas personas experimentan cambios en su habilidad de desarrollar y seguir un plan o trabajar con n\'umeros. Pueden tener dificultad en seguir una receta conocida o manejar las cuentas mensuales. Pueden tener problemas en concentrarse y les puede costar m\'as tiempo hacer cosas ahora que antes. & Hacer errores de vez en cuando al sumar y restar. \\ \hline
			
			A las personas que padecen de Alzheimer muy a menudo se les hace dif\'icil completar tareas cotidianas. A veces pueden tener dificultad en llegar a un lugar conocido, administrar un presupuesto en el trabajo o recordar las reglas de un juego muy conocido. & Necesitar ayuda de vez en cuando para usar el microondas o grabar un programa de televisi\'on. \\ \hline
			
			Olvidar las fechas, estaciones y el paso del tiempo. Pueden tener dificultad en comprender algo si no est\'a en proceso en ese instante. Es posible que se les olvide a veces d\'onde est\'an y c\'omo llegaron all\'i. & Confundirse sobre el d\'ia de la semana pero darse cuenta despu\'es.  \\ \hline
			
			Tener problemas espec\'ficos de la vista como dificultad en leer, juzgar distancias y determinar color o contraste, lo cual puede causar problemas para conducir un veh\'iculo.  & Cambios de la vista relacionados con las cataratas.  \\ \hline
			
			Pueden tener problemas en seguir o participar en una conversaci\'on. Es posible que paren en medio de conversar sin idea de c\'omo seguir o que repitan mucho lo que dicen; que luchen por encontrar las palabras correctas, el vocabulario apropiado o que llamen cosas por un nombre incorrecto (como llamar un "l\'apiz" un "palito para escribir"). & Tener dificultad a veces en encontrar la palabra exacta al hablar. \\ \hline
			
			Una persona con el Alzheimer suele colocar cosas fuera de lugar. Se les puede perder cosas sin poder retrasar sus pasos para poder encontrarlas. A veces, es posible que acusen a los dem\'as de robarles. & De vez en cuando, colocar cosas (como los lentes o el control remoto) en un lugar equivocado. \\ \hline
		\end{tabular}
	\end{center}
	
	\begin{center}
	\begin{tabular}{|p{10cm}|p{7cm}|}
		\hline
		\textbf{S\'intomas de Alzheimer }&\textbf{ Cambios t\'ipicos relacionados con la edad} \\
		\hline
		
		Pueden experimentar cambios en el juicio o en tomar decisiones. Por ejemplo, es posible que regalen grandes cantidades de dinero a las personas que venden productos y servicios por tel\'efono. Puede ser que presten menos atenci\'on al aseo personal.  & Tomar una mala decisi\'on de vez en cuando.  \\ \hline
		
		Empezar a perder la iniciativa para ejercer pasatiempos, actividades sociales, proyectos en el trabajo o deportes. Es posible que tengan dificultad en entender c\'omo ejercer su pasatiempo favorito. Tambi\'en pueden evitar tomar parte en actividades sociales a causa de los cambios que han experimentado. & Estar a veces cansado de las obligaciones del trabajo, de familia y sociales. \\ \hline
		
		El humor y la personalidad de las personas con el Alzheimer pueden cambiar. Pueden llegar a ser confundidas, sospechosas, deprimidas, temerosas o ansiosas. Se pueden enojar f\'acilmente en casa, en el trabajo, con amigos o en lugares donde est\'an fuera de su ambiente.  & Desarrollar maneras muy espec\'ificas de hacer las cosas y enojarse cuando la rutina es interrumpida. \\ \hline
	\end{tabular}
		\begin{center}
			\textbf{Tabla 1:} 10 se\~nales de advertencia de la enfermedad de Alzheimer, Alzheimer’s Association. 
		\end{center}
	\end{center}
	
	
	\subsection{Anexo 2: Tipos de Demencias}
	Al inicio de la enfermedad de Alzheimer, diversos estados patol\'ogicos, presentan s\'intomas a este, para evitar un diagn\'ostico err\'oneo es necesario conocer las caracter\'isticas de cada uno. Algunas de estas enfermedades y sus rasgos distintivos son: \cite{ref6} \\
	\\
	\underline{\textit{Demencia vascular}} es una disminuci\'on en las habilidades de pensamiento causada por condiciones que bloquean o reducen el flujo de sangre al cerebro, privando a las c\'elulas cerebrales del ox\'igeno y los nutrientes vitales. Estos cambios en las habilidades de pensamiento a veces se producen repentinamente despu\'es de un derrame cerebral que bloquea los vasos sangu\'ineos principales del cerebro. Se considera ampliamente la segunda causa m\'as com\'un de demencia despu\'es de la enfermedad de Alzheimer. \cite{ref9} \\
	\\
	\underline{\textit{Demencia mixta}} es una condici\'on en la cual ocurren anormalidades propias de m\'as de un tipo de demencia simult\'aneamente. Los s\'intomas pueden variar, dependiendo de los tipos de cambios cerebrales involucrados y las regiones del cerebro afectadas, y pueden ser similares o incluso indistinguibles de los de la enfermedad de Alzheimer u otra demencia. \cite{ref9} \\
	\\
	\underline{\textit{La demencia de la enfermedad de Parkinson}} es una alteraci\'on en el pensamiento y en el razonamiento que finalmente afecta a muchas personas con la enfermedad de Parkinson. A medida que los cambios cerebrales se extienden gradualmente comienzan, a menudo, a afectar las funciones mentales que incluyen la memorizaci\'on, la capacidad de prestar atenci\'on, el discernimiento y la planificaci\'on de los pasos necesarios para completar una tarea. \cite{ref9} \\
	\\
	\underline{\textit{Demencia con cuerpos de Lewy}} es un tipo de demencia progresiva que conduce a la disminuci\'on del pensamiento, el razonamiento y la funci\'on independiente debido a los dep\'ositos microsc\'opicos anormales que da\~nan las c\'elulas cerebrales. \cite{ref9} \\
	\\
	\underline{\textit{La demencia de la enfermedad de Huntington}} es un trastorno cerebral progresivo causado por un gen defectuoso. Provoca cambios en el \'area central del cerebro, que afectan el movimiento, el estado de \'animo y las habilidades de pensamiento. \cite{ref9} \\
	\\
	\underline{\textit{La enfermedad de Creutzfeldt-Jakob}} es la forma humana m\'as com\'un de un grupo de trastornos cerebrales poco comunes y fatales conocidos como enfermedades pri\'onicas. La prote\'ina pri\'onica mal plegada destruye las c\'elulas cerebrales lo que ocasiona un da\~no que conduce a una r\'apida disminuci\'on del pensamiento y del razonamiento, como as\'i tambi\'en, a movimientos musculares involuntarios, confusi\'on, dificultad para caminar y cambios de humor. \cite{ref9} \\
	\\
	\underline{\textit{La demencia frontotemporal (FTD)}} es un grupo de trastornos causados por la degeneraci\'on progresiva de c\'elulas en los l\'obulos frontales del cerebro (las \'areas ubicadas detr\'as de la frente) o en sus l\'obulos temporales (las regiones ubicadas detr\'as de las orejas). \cite{ref9} \\
	\\
	\underline{\textit{La hidrocefalia de presi\'on normal}} es un trastorno cerebral en el que se acumula exceso de l\'iquido cefalorraqu\'ideo en los ventr\'iculos del cerebro, causando problemas de pensamiento y razonamiento, dificultad para caminar y p\'erdida del control de la vejiga. \cite{ref9} \\
	\\
	\underline{\textit{La demencia del s\'indrome de Down}} se desarrolla en personas que nacen con material gen\'etico adicional del cromosoma 21, uno de los 23 cromosomas humanos. Al igual que las personas con el sindrome de Down envejecen, esas personas tienen un riesgo mucho mayor de desarrollar un tipo de demencia que es el mismo o muy similar a la enfermedad de Alzheimer. \cite{ref9} \\
	\\
	\underline{\textit{El s\'indrome de Korsakoff}} es un trastorno cr\'onico de la memoria causado por la deficiencia grave de tiamina (vitamina B1). Es causado m\'as com\'unmente por el abuso de alcohol, pero otras condiciones tambi\'en pueden causar el s\'indrome. \cite{ref9} \\
	\\
	\underline{\textit{La atrofia cortical posterior (ACP)}} es la degeneraci\'on gradual y progresiva de la capa externa del cerebro (la corteza) situada en la parte posterior de la cabeza. No se sabe si la ACP es una enfermedad \'unica o una posible variante de la enfermedad de Alzheimer. \cite{ref9}
	
	
	\bibliographystyle{plain}
	\bibliography{bibfile}
	
\end{document}
% end of file template.tex

